\documentclass[a4paper,10pt]{article}
\usepackage[utf8]{inputenc}
%\usepackage[T1]{fontenc}
\usepackage{hyperref}
\usepackage{graphicx}
\graphicspath{{./images}}

\hypersetup{
    colorlinks=true,
    linkcolor=blue,
    filecolor=magenta,      
    urlcolor=cyan,
}

\urlstyle{same}

%opening
\title{Saobraćajne nesreće u Francuskoj od 2005. do 2016. godine}
\author{Jovan Ležaja \\ 
	473/2018 \\
	Matematički fakultet, Beograd \\
	navoj96@gmail.com \\
	\\
	Aleksandar Vračarević \\
	434/2016 \\
	Matematički fakultet, Beograd \\
	vracarevicaleksandar@gmail.com}

\begin{document}

\maketitle

\section{Uvod}
Ovaj rad se fokusira na analizu skupa podataka o saobraćajnim nesrećama u Francuskoj od 2005. do 2006. godine. Pozabavićemo se opisom, 
analizom i pretprocesiranjem datih podataka, a potom ćemo različitim algoritmima pokušati da pronadjemo pravila pridruživanja 
(eng. \textit{Association rules}) koristeći se alatima koje nudi IBM SPSS Modeler.

\section{Opis podataka}

Podaci su preuzeti sa \href{https://www.kaggle.com/ahmedlahlou/accidents-in-france-from-2005-to-2016}{https://www.kaggle.com/ahmedlahlou/accidents-in-france-from-2005-to-2016}
i predstavljaju podatke o saobraćajnim nesrećama u Francuskoj prikupljene u period od 2005. do 2016. godine. Kako bismo uopšte pristupili istraživanju skrivenih pravila u okviru ovog skupa, najpre se moramo upoznati sa istim. Naime, skup se sastoji od 5 tabela u \texttt{.csv} formatu. U nastavku ćemo opisati atribute svake od njih. \\



\begin{itemize}
 \item \texttt{caracteristics.csv}
 \begin{itemize}
  \item \textbf{Num\_Acc} : identifikator nesreće - numerički
  \item \textbf{jour} : dan u mesecu - numerički [1-31]
  \item \textbf{mois} : mesec - numerički [1-12]
  \item \textbf{an} : poslednje dve cifre godine - numerički [5-16]
  \item \textbf{hrmn} : vreme u formatu (ssmm) - numerički [1-2.36k]
  \item \textbf{lum} : osvetljenje u trenutku nesreće brojevi [1-5] kodirani na sledeći način:
			\begin{itemize}
			 \item 1 - dan
			 \item 2 - sumrak/zora
			 \item 3 - noć bez prisutnog javnog osvetljenja
			 \item 4 - noć sa isključenim javnim osvetljenjem
			 \item 5 - noć sa uključenim javnim osvetljenjem
			\end{itemize}
  \item \textbf{dep} : INSEE kod odeljenja praćen nulom %TODO: opisati bolje
  \item \textbf{com} : kod opštine izdat od strane INSEE
  \item \textbf{agg} : \begin{itemize} %TODO: 
                        \item 1 - izvan gradske sredine
                        \item 2 - unutar gradske sredine
                       \end{itemize}
  \item \textbf{int} : tip raskrsnice [1-9] kodirani na sledeći način:
			\begin{itemize}
			 \item 1 - van raskrsnice
			 \item 2 - X raskrsnica
			 \item 3 - T raskrsnica
			 \item 4 - Y raskrsnica
			 \item 5 - raskrsnica sa više od 4 kraka
			 \item 6 - kružni tok
			 \item 7 - place %TODO: sta je ovo
			 \item 8 - pružni prelaz
			 \item 9 - ostalo
			 
			\end{itemize}

  \item \textbf{atm} : atmosferski uslovi [1-9] kodirani na sledeći način:
			\begin{itemize}
			 \item 1 - normalni
			 \item 2 - slaba kiša
			 \item 3 - jaka kiša 
			 \item 4 - sneg/gr\^ad
			 \item 5 - magla/dim
			 \item 6 - jak vetar/oluja
			 \item 7 - zaslepljujuće vreme %TODO: š t a
			 \item 8 - oblačno
			 \item 9 - ostalo
			\end{itemize}
			
  \item \textbf{col} : tip sudara [1-7] kodiran na sledeći način:
			\begin{itemize}
			 \item 1 - čeoni sudar
			 \item 2 - sudar otpozadi
			 \item 3 - sudar sa strane
			 \item 4 - lančani sudar
			 \item 5 - višestruki sudari (više vozila i više sudara)
			 \item 6 - drugi sudari
			 \item 7 - nesreća bez sudara
			\end{itemize}
  
  \item \textbf{adr} : poštanska adresa - niska (popunjava se samo za gradske sredine)
  \item \textbf{gps} : GPS kod - jedan karakter:
			\begin{itemize}
			 \item M - Métropole
			 \item A - Antilles (Martinique or Guadeloupe)
			 \item G = Guyane
			 \item R = Réunion
			 \item Y = Mayotte
			\end{itemize}
			
  \item \textbf{lat} : geografska širina izražena u broju stepeni
  \item \textbf{long} : geografska dužina izražena u broju stepeni



 \end{itemize}

 \item \texttt{holidays.csv}
 \begin{itemize}
  \item \textbf{ds} : datum nesreće u formatu godina-mesec-dan
  \item \textbf{holiday} : naziv praznika
 \end{itemize}
 
 \item \texttt{places.csv}
 \begin{itemize}
  \item \textbf{Num\_Acc} : identifikator nesreće - numerički
  \item \textbf{catr} : kategorija puta [1-9] kodirani na sledeći način:
	\begin{itemize}
	 \item 1 - autoput
	 \item 2 - državni put
	 \item 3 - departmentalni putevi
	 \item 4 - komunalni putevi
	 \item 5 - mreža puteva zabranjena za javnost
	 \item 6 - javni parking
	 \item 9 - ostalo
	\end{itemize}
  \item \textbf{voie} : broj puta - numerički
  \item \textbf{V1} : numerički indeks broja puta (na primer: 2 bis, 3 ter itd.)
  \item \textbf{V2} : alfanumerički indeks puta
  \item \textbf{circ} : tip saobraćanja [1-4] kodiran na sledeći način:
	\begin{itemize}
	 \item 1 - jednosmerna ulica
	 \item 2 - dvosmerna ulica
	 \item 3 - razdvojen kolovoz
	 \item 4 - 
	\end{itemize}
  \item \textbf{nbv} : ukupan broj traka na putu - numerički
  \item \textbf{vosp} : indikator postojanja rezervisane trake [1-3], 
			nezavisno od toga da li se nesreća dogodila u toj traci, kodiran na sledeći način:
	\begin{itemize}
	 \item 1 - bickilistička traka
	 \item 2 - parking za bicikle
	 \item 3 - rezervisan kanal
	\end{itemize}
  \item \textbf{prof} : kategorije puta [1-4] zavisno od nagiba puta, kodirane na sledeći:
	\begin{itemize}
	 \item 1 - ``dish''
	 \item 2 - nizbrdica
	 \item 3 - vrh brda
	 \item 4 - dno brda
	\end{itemize}
  \item \textbf{pr} : PR broj kuće - numerička vrednost
  \item \textbf{pr1} : udaljenost od najbližeg PR broja izražena u metrima - numerička vrednost
  \item \textbf{plan} : izgled puta na mapi [1-4], kodirano na sledeći način:
	\begin{itemize}
	 \item 1 - prav put
	 \item 2 - zakrivljen ulevo
	 \item 3 - zakrivljen udesno
	 \item 4 - ``S'' oblika
	\end{itemize}
  \item \textbf{lartpc} : širina ostrva na ulici, ako postoji - niska
  \item \textbf{larrout} : širina puta namenjena za saobraćaj - niska
  \item \textbf{surf} : stanje terena [1-9], kodiran na sledeći način:
	\begin{itemize}
	 \item 1 - normalan
	 \item 2 - vlažan
	 \item 3 - teren
	 \item 4 - potopljen
	 \item 5 - sneg na terenu
	 \item 6 - blatnjav
	 \item 7 - poledica na terenu
	 \item 8 - masan/zauljen teren
	 \item 9 - ostalo
	\end{itemize}
  \item \textbf{infra} : infrastruktura puteva [1-7], kodirana na sledeći način:
	\begin{itemize}
	 \item 1 - podzemni tunel
	 \item 2 - most/nadvožnjak
	 \item 3 - uključenje
	 \item 4 - pruga
	 \item 5 - ``carrefour arranged''
	 \item 6 - pešačka zona
	 \item 7 - ostalo
	\end{itemize}
  \item \textbf{situ} : pozicija nesreće [1-5], kodirana na sledeći način:
	\begin{itemize}
	 \item 1 - na putu
	 \item 2 - u zaustavnoj traci
	 \item 3 - na ivičnjaku
	 \item 4 - na trotoaru
	 \item 5 - na biciklističkoj stazi
	\end{itemize}
  \item \textbf{env1} : locirano blizu škole - numerička vrednost
 \end{itemize}

 \item \texttt{users.csv}
 \begin{itemize}
  \item \textbf{Acc\_number} : identifikator nesreće - numerički
  \item \textbf{Num\_Veh} : identifikator vozila - alfanumerički
  \item \textbf{place} : pozicija osobe u vozilu u vreme nesreće, kodirano u skladu sa sledećom slikom:
  
\begin{minipage}{0.7\textwidth}
 \centering
 \makebox[\textwidth][c]{\includegraphics[width=\textwidth]{./images/ip_sem_img_4.png}}
\end{minipage}
  
  \item \textbf{catu} : uloga osobe u saobraćaju u trenutku nesreće [1-4], kodirano na sledeći način:
	\begin{itemize}
	 \item 1 - vozač
	 \item 2 - putnik
	 \item 3 - pešak
	 \item 4 - pešak na rolerima ili skuteru
	\end{itemize}
  \item \textbf{grav} : ozbiljnost povrede [1-4], kodirana na sledeći način:
	\begin{itemize}
	 \item 1 - neozledjen
	 \item 2 - ubijen
	 \item 3 - hospitalizovan
	 \item 4 - blaga ozleda
	\end{itemize}
  \item \textbf{sex} : pol osobe:
	\begin{itemize}
	 \item 1 - muško
	 \item 2 - žensko
	\end{itemize}
  \item \textbf{Year\_on} : godina rodjenja - numerički
  \item \textbf{trip} : razlog putovanja [1-9], kodiran na sledeći način:
	\begin{itemize}
	 \item 1 - kuća-posao
	 \item 2 - posao-kuća
	 \item 3 - kupovina
	 \item 4 - poslovni put
	 \item 5 - razonoda
	 \item 9 - ostalo
	\end{itemize}
  \item \textbf{secu} : niska koja se sastoji od 2 broja. 
			Prvi označava postojanje sigurnosne opreme [1-9], kodirano na sledeći način:
	\begin{itemize}
	 \item 1 - pojas za vezivanje
	 \item 2 - kaciga
	 \item 3 - sedeljka za decu
	 \item 4 - reflektujuća oprema
	 \item 9 - ostalo
	\end{itemize}
	
	Drugi označava korišćenje sigurnosne opreme [1-3], kodirano na sledeći način:
	\begin{itemize}
	 \item 1 - oprema je korišćena
	 \item 2 - oprema nije korišćena
	 \item 3 - neodredjeno
	\end{itemize}

  \item \textbf{locp} : pozicija pešaka [1-8], kodirano na sledeći način:
	\begin{itemize}
	 \item 1 - više od 50 metara od pešačkog prelaza
	 \item 2 - manje od 50 metara od pešačkog prelaza
	 \item 3 - na pešačkom prelazu sa semaforom
	 \item 4 - na pešačkom prelazu bez semafora
	 \item 5 - na trotoaru
	 \item 6 - na ivičnjaku
	 \item 7 - pod zaklonom
	 \item 8 - u prolazu
	\end{itemize}

  \item \textbf{actp} : akcija pešaka [0-9], kodirano na sledeći način:
	\begin{itemize}
	 \item 0 - neodredjeno
	 \item 1 - kreće se u istom smeru kao i vozilo sa kojim se dogodio sudar
	 \item 2 - kreće se u suprotnom smeru kao i vozilo sa kojim se dogodio sudar
	 \item 3 - prelazak ulice
	 \item 4 - zaklonjen
	 \item 5 - u trku
	 \item 6 - sa životinjom
	 \item 9 - ostalo
	\end{itemize}

  \item \textbf{etatp} : kategorička vrednost koja odredjuje da li je pešak bio u društvu drugih ljudi ili ne, kodirano na sledeći način:
	\begin{itemize}
	 \item 1 - sam
	 \item 2 - sa saputnikom
	 \item 3 - u grupi ljudi
	\end{itemize}

 \end{itemize}
 
 \item \texttt{vehicles.csv}
 \begin{itemize}
  \item \textbf{Num\_Acc} : identifikator nesreće - numerički
  \item \textbf{Num\_veh} : identifikator vozila - alfanumerički kod
  \item \textbf{GP} : %TODO: š t a
  \item \textbf{CATV} : kategorija vozila [01 - 13]
		      \begin{itemize}
		       \item 01 - bicikl
		       \item 02 - moped < 50 kubika
		       \item 03 - kvadricikl sa motorom
		       \item 04 - suvišno od 2006. (registrovani skuter)
		       \item 05 - suvišno od 2006. (motocikl)
		       \item 06 - suvišno od 2006. (putnička prikolica za motocikl)
		       \item 07 - VL %TODO: tebra
		       \item 08 - neupotrebljena kategorija (VL i karavan)
		       \item 09 - neupotrebljena kategorija (VL i prikolica)
		       \item 10 - VU %TODO: nmg
		       \item 11 - najviše korišćeno posle 2006. godine (VU(10) + karavan)
		       \item 12 - najviše korišćeno posle 2006. godine (VU(10) + prikolica)
		       \item 13 - PL samo 3.5T
		       \item 14 - 
		      \end{itemize}

 \end{itemize}
 
\end{itemize}

\section{Analiza i pretprocesiranje podataka}
Prilikom učitavanja tabele \textit{characteristics} smo uočili da je usled loše formatirane datoteke došlo do pogrešne reprezentacije podataka, što smo razrešili jednostavnom \textit{Python} skriptom.
Analizirajući tabelu \texttt{characteristics} uočili smo da atributi \textit{gps}, \textit{lat} i \textit{long} imaju značajan broj 
nedostajućih vrednosti (preko 50\%), a s obzirom da zamena nekom konkretnom vrednošću nema smisla zato što nemamo dovoljno validnih vrednosti u koloni da njihova zamena bude smislena,
odlučili smo da ih uklonimo, jer smatramo da nam nisu bitni za dalju analizu. Kada je reč o atributima \textit{atm} i \textit{col}, 
zbog izuzetno malog broja nedostajućih vrednosti (atributi su bili kompletni blizu 100\%), u čvoru \textit{Type} smo ih odbacili, 
jer ne gubimo ništa odbacivanjem tako malog broja podataka. U tabeli se isto tako nalaze i atributi vezani za lokaciju nesreća (ulica, opština, itd.),
ali dodatnim posmatranjem smo primetili da je format zapisa tih podataka dosta nekonzistentan, tako da je njihova korisnost dovedena u pitanje,
pošto bez iscrpnog analiziranja teksta ne bismo mogli da izvučemo korisne informacije, što je dovelo do odluke da preko čvora \textit{Type} 
tim atributima postavimo ulogu (``Role'') na vrednost \texttt{None}.

\begin{figure}[h!]
 \centering
 \makebox[\textwidth][c]{\includegraphics[width=1.4\textwidth,height=0.25\textheight]{./images/ip_sem_img_1.png}}
 \caption{Sadržaj \textit{Data Audit} čvora za tabelu \texttt{characteristics}}
\end{figure}

\begin{figure}[h!]
 \centering
 \makebox[\textwidth][c]{\includegraphics[width=0.55\textwidth]{./images/ip_sem_img_2.png}}
 \caption{Sadržaj \textit{Filter} čvora za tabelu \texttt{characteristics}}
\end{figure}

\begin{figure}[h!]
 \centering
 \makebox[\textwidth][c]{\includegraphics[width=1.3\textwidth,height=0.5\textheight]{./images/ip_sem_img_3.png}}
 \caption{Sadržaj \textit{Type} čvora za tabelu \texttt{characteristics}}
\end{figure}

Analizom skupa podataka \texttt{users} uočili smo da atributi \textit{locp}, \textit{actp} i \textit{etatp}, koji predstavljaju informacije vezane
za pešaka, imaju značajan broj neodredjenih vrednosti (preko 50\%), tako da smo te kolone izbacili iz skupa podataka \textit{users}. Kada je u pitanju
atribut \textit{secu}, čije su vrednosti predstavljene kao dva broja, uočili smo nekonzistentnost odredjenih polja sa zadatim opisom reprezentacije tog atributa,
tako da smo te nekonzistentne vrednosti preimenovali u \texttt{NA} (neodredjenu vrednost). Za svaki atribut koji je imao 0 kao vrednost, a nije bilo definisano
šta ta vrednost predstavlja, 0 je zamenjena sa \texttt{NA}. Za atribut \textit{place} smo sve vrednosti ostavili kakve jesu, pošto je šema 
koja predstavlja kodiranje bila nedovoljno jasna.

Skup podataka \texttt{vehicles} smo analizirali i zaključili da sve slogove koji sadrže nedostajuće i neodredjene 
vrednosti možemo da odbacimo. Nismo naišli ni na kakve nepravilnosti koje iziskuju podrobnije procesiranje.

Nakon što smo uvideli da kolone \texttt{v1} i \texttt{v2} skupa \texttt{places} sadrže ogroman broj nedostajućih
vrednosti, odbacili smo ih. Pošto se u kolonama \texttt{pr} i \texttt{pr1} javlja preko 50\% nedostajućih vrednosti, a smatramo da ne postoji
smislen način da te vrednosti popunimo, odbacili smo i ove kolone. Kolona \texttt{env1} predstavlja predstavlja meru blizine
školi, ali je zbog nejasnog kodiranja i ova kolona odbačena. Iako kolone \texttt{voie}, \texttt{vosp}, \texttt{lartpc}, \texttt{infra}
i \texttt{nbv} nemaju puno nedostajućih vrednosti, gotovo svi slogovi uzimaju mali skup vrednosti za pomenute atribute pa smo 
se odlučili da ni ove atribute ne koristimo u daljoj analizi. Za kolone \texttt{situ}, \texttt{prof}, \texttt{surf} i \texttt{plan}
ćemo odbaciti slogove sa vrednošću nula za ove atribute. U koloni \texttt{larrout} se javljaju negativne vrednosti za širinu
puta, pa ćemo i njih ukloniti.

Skup podataka \texttt{holidays} smo odlučili da ne koristimo za dalju analizu, jer ne sadrži preterano korisne informacije.

\section{Pravila pridruživanja}
Nakon što smo obradili skupove podataka, hteli smo da na svaki od relevantnih skupova primenimo algoritme \textit{Apriori} i \textit{Carma},
u nadi da ćemo uočiti neka zanimljiva pravila. Nakon primene pomenutih algoritama, cilj nam je bio da primenimo iste algoritme nad objedinjenim podacima.

\subsection{Primena \textit{Apriori} i \textit{Carma} algoritama nad skupom \texttt{characteristics}}
Iskoristili smo niz čvorova \textit{Reclassify} kako bismo lakše tumačili kategoričke vrednosti. Iz tog skupa smo filtrirali one slogove čija je 
vrednost atributa \textit{tip\_raskrsnice} \texttt{NA}. Potom smo primenili \textit{Apriori} sa podrazumevanim podešavanjima 
(minimalna podrška uzročnika je 10\%, a minimalna pouzdanost pravila je 80\%) i rezultati izvršavanja
tog algoritma se mogu videti na slici. Posledice pravila se odnose na atmosferske prilike, tip raskrsnice i indikator da li se nesreća desila u gradu ili ne.
Algoritam je uspeo da nadje 30 pravila, koja sve u svemu nisu zanimljiva. Naime, lift mera se kreće u opsegu od 0.998 do 1.314, što nam govori da su
uzročnici u blagoj korelaciji sa posledicama. Kako bismo pripremili skup podataka za algoritam \textit{Carma}, koristili smo čvor \textit{SetToFlag}. 
\textit{Carma} algoritam smo primenili sa istim parametrima kao i \textit{Apriori}. Dobili smo 22 pravila, koja su gotovo identična onima koje smo
dobili korišćenjem \textit{Apriori} algoritma. \\
% TODO: slika Apriori(5) i Carma(6)
 
Kao što se iz rezultata \textit{Data Audit} čvora može primetiti, odredjene vrednosti nekh atributa dominiraju nad ostalim vrednostima, te stoga ne
čudi što otkrivena pravila sadrže te vrednosti. 
%TODO: slika 7+

\begin{figure}[!h]
 \centering
 \label{fig:data_audit}
 \makebox[\textwidth][c]{\includegraphics[width=1\textwidth,height=0.3\textheight]{./images/stream1_data_audit_hist.PNG}}
 \caption{Možemo videti da se u kolonama \textit{osvetljenje}, \textit{tip\_raskrsnice} i \textit{atmosferske\_prilike} u 
 najvećem broju slučajeva javlja samo jedna vrednost, te ćemo te vrednosti pokušati da izbalansiramo. Još jedna kolona na koju ćemo 
 primeniti balansiranje je \textit{vrsta\_sudara}.}
\end{figure}

U želji da izbor pravila bude pravedniji, odlučili smo da izbalansiramo skup podataka, tako što 
ćemo korišćenjem čvorova \textit{Balance} na pojedinačne kolone ublažiti efekat dominantnih vrednosti (pomenute čvorove
smo generisali uz pomoć distribucija odgovarajućih kolona). Nakon toga smo redom primenjivali \textit{Apriori}
algoritam za svaku izmenjenu kolonu. Potom smo eksperimentisali sa primenom \textit{Apriori} alogritma na ulančane \textit{Balance} čvorove. 
Neki od rezultata su predstavljeni na narednim slikama.

\begin{figure}[!h]
 \centering
 \makebox[\textwidth][c]{\includegraphics[width=1.3\textwidth,height=0.2\textheight]{./images/stream1_apriori_bal_osvetljenje.PNG}}
 \caption{Rezultat primene apriori algoritma na balansiranu kolonu \textit{osvetljenje}. Izdvojena pravila jesu logična, ali nam 
 ne otkrivaju puno interesantnih zaključaka.}
\end{figure}

\begin{figure}[h!]
 \centering
 \makebox[\textwidth][c]{\includegraphics[width=1.3\textwidth,height=0.4\textheight]{./images/stream1_apriori_red_sudar.PNG}}
 \caption{Rezultat primene apriori algoritma na balansiranu kolonu \textit{vrsta\_sudara}. U ovom slučaju je pronadjeno 33 pravila
  od kojih su ona sa najvećom lift merom logična ali i dalje nam ne daju upotrebljiviji uvid u zavisnosti medju atributima. }
\end{figure}

\clearpage
Zaključujemo da balansiranje pojedinačnih kolona ne dovodi do željenih rezultata te smo probali sa ulančanim balansiranjem.

\begin{figure}[!h]
 \centering
 \makebox[\textwidth][c]{\includegraphics[width=1\textwidth,height=0.1\textheight]{./images/stream1_apriori_bal_osvetljenje_SEQ.PNG}}
 \caption{Rezultat primene apriori algoritma na redom izbalansirane sve kolone skupa. Izdvojeno pravilo prema lift meri jeste
 zanimljivo ali je opet očekivano da u gradskoj sredini postoji osvetljenje koje je uključeno. }
\end{figure}

%stream1_apriori_reduce_atm_SEQ.PNG
\begin{figure}[!h]
 \centering
 \makebox[\textwidth][c]{\includegraphics[width=1\textwidth,height=0.1\textheight]{./images/stream1_apriori_reduce_atm_SEQ.PNG}}
 \caption{Rezultat primene apriori algoritma na balansirane kolone \textit{tip\_raskrsnice}, \textit{vrsta\_sudara} 
 i \textit{atmosferske\_prilike}. }
\end{figure}


Kako pokušaj sa balansiranjem nije prošao slavno, odlučili smo se da potpuno eliminišemo slogove koji imaju najzastupljeniju vrednost odredjenog atributa,
i da na njega primenimo iste algoritme. Na ovaj način smo otkrili više pravila nego u prethodnim pokušajima, sa lift merama 
u opsegu od 0.673 do 1.421, ali sa malom pouzdanošću i podrškom. Rezultati se mogu videti na slici.

\begin{figure}[!h]
 \centering
 \makebox[\textwidth][c]{\includegraphics[width=1.1\textwidth,height=0.3\textheight]{./images/stream1_apriori_vrsta_sudara.PNG}}
 \caption{Nakon izbacivanja slogova koji bi `prigušili` ostatak skupa, dobijeni su ovakvi rezultati. I dalje smatramo da ne 
 postoje izuzetno zanimljiva pravila. }
\end{figure}

\clearpage

\begin{figure}[h!]
 \centering
 \makebox[\textwidth][c]{\includegraphics[width=1.3\textwidth,height=0.4\textheight]{./images/stream1_model.PNG}}
 \caption{Prikaz celokupnog streama obrade skupa \textit{caracteristics.csv} }
\end{figure}

%TODO: 	broj putnika: nesto slogirano, brisemo
% 	

\subsection{Primena \textit{Apriori} i \textit{Carma} algoritama nad skupom \texttt{places}}

\subsection{Primena \textit{Apriori} i \textit{Carma} algoritama nad objedinjenim podacima}

Uz pomoć čvora \textit{Merge} smo spojili skupove \textit{users}, \textit{vehicles} i \textit{caracteristics} izbacujući 
redove sa nedostajućim vrednostima usput. Nakon toga smo izbacili kolone ... Izvršili smo dodatno uklanjanje besmislenih slogova
i tako pripremljene podatke propustili kroz \textit{Apriori} i \textit{Carma} čvorove. U prvoj iteraciji smo koristili
podrazumevane parametre za oba čvora. Rezultati su u nastavku.

\begin{figure}[h!]
 \centering
 \makebox[\textwidth][c]{\includegraphics[width=1.3\textwidth,height=0.4\textheight]{./images/default_apriori.PNG}}
 \caption{Rezultati apriori algoritma sa podrazumevanim parametrima.}
\end{figure}

\begin{figure}[h!]
 \centering
 \makebox[\textwidth][c]{\includegraphics[width=1.3\textwidth,height=0.4\textheight]{./images/default_carma.PNG}}
 \caption{Rezultati carma algoritma sa podrazumevanim parametrima.}
\end{figure}

Primećujemo da su prisutna pravila sa solidnom lift merom od oko 1.5 i velikom podrškom za oba algoritma. Odmah se Primećuje
i ogromna razlika u količini pronadjenih pravila. Većina pravila pronadjenih aprior algoritmom kao posledicu imaju nošenje
zaštitnog pojasa u trenutku nesreće. Kod carma algoritma pravila su malo raznolikija ali i dalje se u pravilima javljaju 
većinski dominantne vrednosti odgovarajućih kolona. U nastavku ćemo pokušati da ispitamo vezu između nekih atributa za koje
mislimo da mogu proizvesti interesantna pravila.

\end{document}
